\documentclass{article}
\usepackage[utf8]{inputenc}
\usepackage{fullpage}
\usepackage{amsfonts, amsmath, pifont}
\usepackage{amsthm}
\usepackage{graphicx}
\usepackage{float}
\usepackage{tkz-euclide}
\usepackage{tikz}
\begin{document}
\begin{center}
	\tikzset{%
		block/.style    = {draw, thick, rectangle, minimum height = 3em,
			minimum width = 3em},
		sum/.style      = {draw, circle, node distance = 2cm}, % Adder
		input/.style    = {coordinate}, % Input
		output/.style   = {coordinate} % Output
	}
	% Defining string as labels of certain blocks.
	\newcommand{\suma}{\Large$+$}
	\newcommand{\inte}{$\displaystyle \int$}
	\newcommand{\derv}{\huge$\frac{d}{dt}$}
	
	\begin{tikzpicture}[auto, thick, node distance=2cm, >=triangle 45]
	\draw
	% Drawing the blocks of first filter :
	node at (0, 0) [input] (inp) {\Large \textopenbullet}
	node [sum, right of=inp] (sum) {\suma}
	node [block, right of=sum] (int) {\inte}
	node [output, right of=int] (out) {}
	node [output, right of=out] (out2) {\Large \textopenbullet}
	node [output, below of=out] (temp1) {}
	node [output, below of=sum] (temp2) {}
	;
	\draw[->](inp) -- node{$x(t)$} (sum);
	\draw[->](sum) -- (int);
	\draw[-](int) -- (out);
	\draw[->](out) -- node{$y(t)$} (out2);
	\draw[-](out) -- (temp1);
	\draw[->](temp2) -- (sum);
	\draw[-](temp1) -- node{$-5$} (temp2);
	\end{tikzpicture}
\end{center}
\end{document}